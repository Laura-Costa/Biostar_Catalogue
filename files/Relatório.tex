\PassOptionsToPackage{table}{xcolor}
\documentclass[12pt, a4paper]{article}
\usepackage{tkz-base}
\usepackage{tkz-euclide}
\usepackage{tikz}
\usepackage{xcolor}
\usepackage{multirow} 
\usepackage{flafter} 
\usetikzlibrary{arrows.meta}
\usepackage[left=2cm, right=2cm, top=2cm, bottom=2cm]{geometry}
\usepackage[brazil]{babel}
\usepackage[utf8]{inputenc}
\usepackage{color}
\usepackage{indentfirst}
\usetikzlibrary{angles, quotes}
\usepackage{amssymb}
\usepackage{amsmath}
\usepackage{pst-eucl}
\usepackage{tabularx,ragged2e,booktabs}
\usepackage{ragged2e,microtype}
\usepackage{array} % for '\newcolumntype' macro
% Define two new column types:
% (a) for full-width columns:
\newcolumntype{L}{@{} >{\RaggedRight}p{\dimexpr12cm+2\tabcolsep+1\arrayrulewidth\relax} @{}}
% (b) for half-width columns:
\newlength\mylen
\settowidth\mylen{$4.$\space} % amount of hanging indentation
\newcolumntype{P}[1]{>{\RaggedRight\hangafter1\hangindent\mylen}p{#1}}
\usepackage{float}
\usepackage{hyperref}
\usepackage[paper=portrait,pagesize]{typearea}
\usepackage{calrsfs}
\usepackage{mathtools}
\newcommand\myeq{\stackrel{\mathclap{\tiny\mbox{Bayes}}}{=}}
\usepackage{hyperref}
\usepackage{xurl}
\newcommand{\stackwords}[2]{\begin{tabular}[t]{@{}l@{}}#1\\#2\end{tabular}}
\begin{document}
	
	\begin{titlepage}
		\begin{center}
			{\large \textbf{Universidade Federal do Rio de Janeiro}}
			
			\vspace{0.2cm}
			
			{\large \textbf{Curso de Ciência da Computação}}
			
			\vspace{6.08cm}%%%%%%%%%%%%%%%%%%%%%%%%%%%%%%%%
			
			
			{\large \textbf{Integração Numérica}}\\
			{\normalsize \textbf{Trabalho para a disciplina de Introdução à Computação Numérica}}
			
			\vspace{6.08cm}%%%%%%%%%%%%%%%%%%%%%%%%%%%%%%%%
			\begin{tabbing}
				\hspace{7cm}\textbf{Alunas: \stackwords{Helena Serrano Cardoso da Costa}{Laura Serrano Cardoso da Costa}}
			\end{tabbing}
			
			\vspace{6.08cm}%%%%%%%%%%%%%%%%%%%%%%%%%%%%%%%%
			
			\textbf{Rio de Janeiro, janeiro de 2024}
		\end{center}	
	\end{titlepage}
	
	\newpage
	
	%\tableofcontents
	
	\newpage
	
	\section*{Comentários sobre o código}
	
	\subsubsection*{Tarefa 1 - item b(i) - Regra dos Trapézios sem lista de valores}
	
	\includegraphics[scale=0.35]{fig1.png}
	
    Desigualdades entre números do tipo {\ttfamily float} que carregam algum tipo
	de erro de arredondamento podem gerar inconsistências em relação
	ao que é esperado. Logo, devido ao acúmulo de erros de arredondamento
	pelo computador, o último ponto considerado pode ficar levemente
	maior do que b. Por isso, no laço de repetição {\ttfamily while} da figura acima, utilizamos a
	condição $i <= n$ em vez da condição $x <= b$ para evitar a perda
	de pontos. Como $i$ e $n$ são números inteiros, os erros de
	arredondamento não interferem nessa desigualdade.
	
	\vspace{1cm}
	
	\includegraphics[scale=0.35]{fig2.png}
	
	Se fizéssemos, no trecho de código destacado acima, $x = x + h$, em vez de $x = a + i \cdot h$, o programa realizaria muitas somas, o que
	 implicaria em muitos erros de arredondamento. Quando calculamos com vários pontos e acrescentamos $h$ muitas vezes, estes erros se acumulam e começam a influenciar nos últimos dígitos de $x$. 
	 
	 Uma forma de evitar os erros de arredondamento é, em vez de somar $h$ a cada iteração, declarar: $x = a + i \cdot h$. Desta forma, o erro de arredondamento é dado pelo produto $i\cdot h$ e pela soma. Como o erro acontece apenas na iteração atual, ele se mantém pequeno.
	
	Abaixo, uma análise do erro da aproximação para o caso de se usar $x = x + h$ ou $x = a + i\cdot h$:
	
	\includegraphics[scale=0.73]{fig3.png}
	
	\subsubsection*{Tarefa 3 - item a}
	
	\includegraphics[scale=0.35]{fig4.png}
	
	Esta parte do código (mostrada na figura acima) encontra a quantidade mínima $n$ necessária de partições do intervalo de integração para se obter um erro $\varepsilon$ menor ou igual a $10^{-8}$ na aproximação da integral pela Regra dos Trapézios. A figura a seguir mostra a fórmula implementada nesta parte do código:
	
	\includegraphics[scale=0.22]{fig5.jpg}
	
	\includegraphics[scale=0.35]{fig6.png}
	
	Analogamente, esta outra parte do código (mostrada na figura acima) encontra a quantidade mínima $n$ necessária de partições do intervalo de integração para se obter um erro $\varepsilon$ menor ou igual a $10^{-8}$ na aproximação da integral pela Regra de Simpson. Depois de obter a quantidade $n$ de partições necessárias, verificamos se a quantidade mímina de pontos correspondente é par. Caso seja, somamos $1$ à quantidade mínima de pontos. A figura a seguir mostra a fórmula implementada nesta parte do código:
	
	\includegraphics[scale=0.18]{fig7.jpg}
	
	\section*{Cálculo da expressão para o erro da \\Regra dos Trapézios}
	
	\includegraphics[scale=0.7]{fig8.jpg}
	
	\section*{Cálculo da expressão para o erro da \\Regra de Simpson}
	
	\includegraphics[scale=0.7]{fig9.jpg}
	
	\includegraphics[scale=0.7]{fig10.jpg}
	
	\section*{Comparar os resultados obtidos pela \\Regra dos Trapézios e pela Regra de Simpson}
	
	Os resultados obtidos pela Regra de Simpson são melhores pois esta regra utiliza um polinômio interpolador de grau $2$. Como o polinômio de grau dois tem inflexão, ele consegue se adequar melhor à curva que se quer integrar. Por isso, quando comparamos os resultados, a aproximação obtida pela Regra de Simpson tem qualidade maior em relação à aproximação encontrada com a Regra dos Trapézios, que faz interpolação linear. 
	
	Se usarmos a mesma quantidade de pontos, a aproximação obtida pela Regra de Simpson será, em geral, melhor. Podemos perceber isso pelo resultado do item $a$ da Tarefa $3$. Nesta tarefa, para satisfazer à uma mesma tolerância $\varepsilon = 10^{-8}$, a Regra dos Trapézios exige uma quantidade bem maior de pontos, quando comparada à quantidade de pontos exigida pela Regra de Simpson.
	Na figura a seguir, os valores destacados com a mesma cor são a quantidade de pontos usada nas aproximações de uma mesma integral através de regras diferentes.
	\begin{center}
		\includegraphics[scale=0.5]{fig11.png}
	\end{center}


	\newpage
	\section*{Referências}
	
	\vspace{20pt}
	
	[1] \url{https://drive.google.com/file/d/1CGOJw9yA6J3UfgQL7xnTlUddk0_nTpkf/view}
	
	\vspace{5pt}
	
	[2] \url{https://www.w3schools.com/python/ref_math_ceil.asp}
	
	\vspace{5pt}
	
	[3] \url{https://www.derivative-calculator.net/}
	
	
	
	
\end{document}